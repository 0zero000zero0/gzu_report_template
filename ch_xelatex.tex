% 字符编码和语言设置
\usepackage[utf8]{inputenc} % 设置输入文件编码为UTF-8
\usepackage[british,UKenglish]{babel} % 设置文档语言为英国英语,提供拼写检查和断行规则
\usepackage{fontenc} % 设置输出编码,通常与 inputenc 配合使用
% 数学公式和符号
\usepackage{amsmath} % 提供数学公式和符号的增强功能
\usepackage{amssymb} % 提供额外的数学符号,如集合符号
\usepackage{mathtools} % 增强 amsmath 包的功能,提供更多数学功能
% 表格和图形
\usepackage{array} % 提供表格和数组环境的增强功能
\usepackage{booktabs} % 提供更专业的表格格式(例如,\toprule,\midrule,\bottomrule等)
\usepackage{longtable} % 允许跨页的长表格
\usepackage{tabularx} %自动设置表格列宽
\usepackage{tikz} % 强大的绘图工具,能够绘制矢量图形
\usepackage{pgfplots} % 基于tikz,用于绘制高质量的二维和三维图表
\usepackage{graphicx} % 用于插入和处理图形文件(如png, jpg, pdf等)
\usepackage{subcaption} % 提供子图标注的功能,替代旧版subfigure
% 字体和排版
\usepackage{microtype} % 改进排版质量,优化字距和字符间距
\usepackage{lmodern} % 使用现代的拉丁字体,提升排版效果
\usepackage{sectsty} % 自定义章节标题的字体样式
\usepackage{xcolor} % 支持颜色的设置,包括文字和表格颜色
% 引用和参考文献
\usepackage{cite} % 简化文献引用并避免重复引用
\usepackage{hyperref} % 为文档提供超链接功能,包括跳转、书签和目录链接
\usepackage{cleveref} % 增强引用功能,自动管理交叉引用
% 页面布局和样式
\usepackage{geometry} % 用于自定义页面边距和纸张大小
\usepackage{fancyhdr} % 自定义页眉和页脚
\usepackage{multicol} % 用于创建多列布局
\usepackage{setspace} % 设置行间距
\usepackage{todo} % 插入任务列表,方便文档编写过程中添加注释
% 算法和伪代码
\usepackage{algorithm} % 提供算法的环境
\usepackage{algorithmic} % 提供详细的算法步骤定义
% 代码高亮和格式化
\usepackage{listings} % 用于代码高亮和格式化,支持多种编程语言
% 附录和文档结构
\usepackage{appendix} % 添加附录部分
\usepackage{subfiles} % 支持将文档拆分为多个子文件,方便组织和管理
\usepackage{nameref} % 支持命名引用
% 表格和列表
\usepackage{enumitem} % 自定义项目符号和编号列表
\usepackage{multirow} % 用于在表格中创建跨行单元格


\newcommand\tab[1][1cm]{\hspace*{#1}}
\hypersetup{colorlinks=true, linkcolor=blue}
\interfootnotelinepenalty=10000


\newcommand{\cleancode}[1]{\begin{addmargin}[3em]{3em}\texttt{\textcolor{cleanOrange}{#1}}\end{addmargin}}
\newcommand{\cleanstyle}[1]{\text{\textcolor{cleanOrange}{\texttt{#1}}}}


\usepackage[colorinlistoftodos,prependcaption,textsize=footnotesize]{todonotes}
\newcommandx{\commred}[2][1=]{\textcolor{Red}
{\todo[linecolor=red,backgroundcolor=red!25,bordercolor=red,#1]{#2}}}
\newcommandx{\commblue}[2][1=]{\textcolor{Blue}
{\todo[linecolor=blue,backgroundcolor=blue!25,bordercolor=blue,#1]{#2}}}
\newcommandx{\commgreen}[2][1=]{\textcolor{OliveGreen}{\todo[linecolor=OliveGreen,backgroundcolor=OliveGreen!25,bordercolor=OliveGreen,#1]{#2}}}
\newcommandx{\commpurp}[2][1=]{\textcolor{Plum}{\todo[linecolor=Plum,backgroundcolor=Plum!25,bordercolor=Plum,#1]{#2}}}

\def\code#1{{\tt #1}}

\def\note#1{\noindent{\bf [Note: #1]}}

\makeatletter
%% The "\@seccntformat" command is an auxiliary command
%% (see pp. 26f. of 'The LaTeX Companion,' 2nd. ed.)
\def\@seccntformat#1{\@ifundefined{#1@cntformat}%
   {\csname the#1\endcsname\quad}  % default
   {\csname #1@cntformat\endcsname}% enable individual control
}
\let\oldappendix\appendix %% save current definition of \appendix
\renewcommand\appendix{%
    \oldappendix
    \newcommand{\section@cntformat}{\appendixname~\thesection\quad}
}
\makeatother

% -------------------------允许算法跨页-------------
\makeatletter
\newenvironment{breakablealgorithm}
  {% \begin{breakablealgorithm}
   \begin{center}
     \refstepcounter{algorithm}% New algorithm
     \hrule height.8pt depth0pt \kern2pt% \@fs@pre for \@fs@ruled
     \renewcommand{\caption}[2][\relax]{% Make a new \caption
       {\raggedright\textbf{\ALG@name~\thealgorithm} ##2\par}%
       \ifx\relax##1\relax % #1 is \relax
         \addcontentsline{loa}{algorithm}{\protect\numberline{\thealgorithm}##2}%
       \else % #1 is not \relax
         \addcontentsline{loa}{algorithm}{\protect\numberline{\thealgorithm}##1}%
       \fi
       \kern2pt\hrule\kern2pt
     }
  }{% \end{breakablealgorithm}
     \kern2pt\hrule\relax% \@fs@post for \@fs@ruled
   \end{center}
  }
\makeatother


